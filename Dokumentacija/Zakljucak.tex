\chapter{Zaključak i budući rad}
		
	%	\textbf{\textit{dio 2. revizije}}\\
		
		 %\textit{U ovom poglavlju potrebno je napisati osvrt na vrijeme izrade projektnog zadatka, koji su tehnički izazovi prepoznati, jesu li riješeni ili kako bi mogli biti riješeni, koja su znanja stečena pri izradi projekta, koja bi znanja bila posebno potrebna za brže i kvalitetnije ostvarenje projekta i koje bi bile perspektive za nastavak rada u projektnoj grupi.}
		
		% \textit{Potrebno je točno popisati funkcionalnosti koje nisu implementirane u ostvarenoj aplikaciji.}
		\normalsize
		Projekt \textit{Papirni.ca} bio je u izradi 4.10.2020. - 14.1.2021. Tijekom tih 14 tjedana tim je izradio aplikaciju s (gotovo svim) traženim funkcionalnostima.
		Zadatak je bio dizajnirati, programirati i isporučiti aplikaciju koja predstavlja papirnicu tj. trgovinu papirom i nekim vezanim uslugama. Korisnici (kupci) podijeljeni su na 2 vrste - privatni i poslovni te ovisno o vrsti kupca mogli su birati između kupovanja papira preddefiniranih i posebnih stilova te pretplaćivati se na proizvode iz papirnice. Plaćanje se trebalo odvijati putem usluga \textit{PayPal} no zbog manjka vremena za taj dio i problema pri implementaciji te funkcionalnosti on nije napravljen (umjesto mjesečnog naplaćivanja, pretplatnike se obavještava o izvršenoj pretplati svaki mjesec).
		\\
		\\
		U prvoj fazi projekta radilo se na okupljanju tima, podijeli posla, dizajnu aplikacije i izradi osnovnih funkcionalnosti. tim je podijeljen na 3 glavne skupine: frontend, backend i miješani poddtim (manji zadaci na frontendu i backendu, dokumentacija). Najprije su osmišljeni izvedba baze podataka i izgled aplikacije. U dokumentaciji su napravljeni prvi osnovni koraci i izrađeni početni dijagrami koji su pomogli u kasnijoj implementaciji funkcionalnosti - obrasci uporabe, sekvencijski dijagrami, model baze
		podataka i dijagram razreda. Tijekom prve faze tim se povremeno sastajao kako bi se razmijenile daljnje ideje i definirali pojedinačni zadaci.  Najveći izazovi u ovoj fazi bili su postavljanje servera i baze podataka na Heroku te uspostavljanje komunikacije između frontenda i backenda koji su razriješeni uz pomoć literature, dokumentacije i već stečenih znanja.
		\\
		\\
		U drugoj fazi projekta radilo se na implementaciji ostatka funkcionalnosti i dovršavanju dokumentacije. Ažurirani su dijagrami iz prethodnih faza (dijagram baze podataka, dijagram razred) uslijed manjih promjena u bazi i nekim razredima. Najveći izazovi u ovoj fazi bili su izrada košarice (cart) i (kasnije) \textit{PayPal}. Backend i frontend timovi imali su regularne tjedne sastanke u ovom periodu na kojemu su osim rasprava zajedno radili na programskom kodu i testirali dobivena rješenja.
		\\
		\\
		Moguće proširenje aplikacije bilo bi uspješno dodavanje usluge \textit{PayPal} i proširenje asortimana papirnice.
		\\
		\\
		Rad na projektu je bilo vrijedno iskustvo i lekcija o zajedničkom radu, organizaciji i usklađenju poddtimova te komunikacije među istima. Proširili smo obzore i stekli nova znanja vezana uz programiranje i rad u sustavu Git koji su zasigurno veliki bonusi za buduće radno okruženje (za koje postoji velika mogućnost da će biti organizirano na sličan način).
		\eject 